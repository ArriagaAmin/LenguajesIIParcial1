\documentclass[a4paper,10pt]{article}

\usepackage[activeacute]{babel}
\usepackage[utf8]{inputenc}
\usepackage{bookman}
\usepackage{color}
\usepackage{graphicx}
\usepackage{anysize}
\usepackage{multicol}
\usepackage[pdftex=true,colorlinks=true,linkcolor=black,urlcolor=blue]{hyperref}

\marginsize{1.5cm}{1.5cm}{1.5cm}{1.5cm}
\newcommand{\HRule}{\rule{\linewidth}{0.5mm}}

\date{}

\pagenumbering{arabic}
\setcounter{page}{1}

\begin{document}

% ===================================== MEMBRETE ===================================== %
\begin{center}
  \textsc {
    Universidad Simón Bolívar \\[0cm]
    Departamento de Computaci\'on y Tecnolog\'ia de la Informaci\'on \\[0cm]
    CI4721 - Lenguajes de Programaci\'on II \\[0cm]
    Trimestre Abril - Julio 2021 \\[0cm]
    Prof. Ricardo Monascal \\[0cm]
    Amin Arriaga 16-10072
  }
  \HRule \\[0.4cm]
  {\Large \textbf{Examen I}} \\[0.2cm]
  \textsc{(20 puntos)}
  \HRule
\end{center}

    El repositorio de Github usado para este examen se encuentr\'a en 
    \href{https://github.com/ArriagaAmin/LenguajesIIParcial1}{este link}.

    \begin{enumerate}
        \item \textit{(6 puntos)} Sea una gram\'atica \verb|EXCEP T = ({instr, ;, try, catch, f inally}, {I}, P, I)|, 
        con \verb|P| definido de la siguiente manera:

        \begin{verbatim}
                        I  -> try I catch I finally I
                            | try I catch I
                            | I ; I
                            | instr
        \end{verbatim}
        Esto representa instrucciones con bloques protegidos (\verb|try|) y manejadores de excepciones 
        (\verb|catch|), que opcionalmente tienen una instrucci\'on que se ejecuta en cualquier caso, ya 
        sea que ocurriera la excepci\'on o no (\verb|finally|). 

        \begin{enumerate}
            \item \textit{(3 puntos)} Aumente la gram\'atica con un nuevo s\'imbolo inicial no recursivo \verb|S|, 
            construya la m\'aquina caracter\'istica \textit{LR(1)} y diga si existen conflictos en el mismo

            \textbf{Respuesta:} Se coloc\'o la nueva gram\'atica, la m\'aquina caracter\'istica y sus conflictos 
            en el siguiente \href{https://github.com/ArriagaAmin/LenguajesIIParcial1/blob/main/Pregunta1/MC_LR1}{archivo} 
            del repositorio.\\

            \item \textit{(1 punto)} Tome en consideraci\'on las siguientes reglas:
            \begin{itemize}
                \item \verb|;| asocia a la izquierda.
                \item \verb|finally| se asocia al \verb|try| m\'as interno.
                \item \verb|finally| tiene m\'as precedencia que \verb|;|.
                \item \verb|catch| tiene m\'as precedencia que \verb|;|
            \end{itemize}
            En caso de haber conflictos en el aut\'omata de prefijos viables \textit{LR(1)}, diga c\'omo 
            resolver\'a los conflictos (seleccionando una de las acciones que conforma dicho conflicto), 
            de tal forma que las reglas anteriores sean satisfechas.

            \textbf{Respuesta:} Los conflictos se resolvieron en el mismo
            \href{https://github.com/ArriagaAmin/LenguajesIIParcial1/blob/main/Pregunta1/MC_LR1}{archivo} 
            de la pregunta anterior.\\

            \item \textit{(2 puntos)} A partir de las respuestas anteriores, construya la m\'aquina 
            caracterı\'istica \textit{LALR(1)} y diga si existen conflictos en el mismo. En caso de 
            existir, explique c\'omo los resolverı\'ia (seleccionando una de las acciones que conforma 
            dicho conflicto), con las mismas reglas de la pregunta anterior.

            \textbf{Respuesta:} Se coloc\'o la m\'aquina caracter\'istica, sus conflictos y la forma de 
            resolverlos en el siguiente \href{https://github.com/ArriagaAmin/LenguajesIIParcial1/blob/main/Pregunta1/MC_LALR1}{archivo} 
            del repositorio.\\
        \end{enumerate}

        \newpage

        \item \textit{(5 puntos)} Considerando la misma gram\'atica de la pregunta anterior:
        
        \begin{enumerate}
            \item \textit{(2 puntos)} Proponga una Relaci\'on de Precedencia de Operadores entre los 
            s\'imbolos terminales de la gram\'atica que permita resolver los conflictos con la misma 
            suposici\'on que en las preguntas anteriores. 
            
            \textbf{Respuesta:}
            \begin{center}
                \begin{tabular}{ | c | c | c | c | c | c | c |}
                    \hline
                    & \verb|try| & \verb|catch| & \verb|finally| & \verb|;| & \verb|instr| & \verb|$| \\ \hline
                    \verb|try| & $<$ & $=$ & & & $<$ & \\ \hline
                    \verb|catch| & $<$ & $>$ & $=$ & $<$ & $<$ & $>$ \\ \hline 
                    \verb|finally| & $<$ & $>$ & $>$ & $>$ & $<$ & $>$ \\ \hline 
                    \verb|;| & $<$ & & & $>$ & $<$ & $>$ \\ \hline 
                    \verb|instr| & & $>$ & $>$ & $>$ & & $>$ \\ \hline 
                    \verb|$| & $<$ & & & $<$ & $<$ & \\ \hline
                \end{tabular}
            \end{center}
            \vspace*{1cm}

            \item \textit{(1.5 puntos)} Use el reconocedor para reconocer la frase:
            \begin{center}
                \verb|instr ; try instr catch instr ; try instr catch instr f inally instr ; instr|
            \end{center}

            \textbf{Respuesta:}
            Primero enumeramos las producciones

            \begin{verbatim}
             (0)    S  -> I $
             (1)    I  -> try I catch I finally I
             (2)        | try I catch I
             (3)        | I ; I
             (4)        | instr
            \end{verbatim}

            \begin{center}
                \begin{tabular}{ | l | l | c |}
                    \hline
                    \textbf{ENTRADA} & \textbf{PILA} & \textbf{ACCI\'ON} \\ \hline

                    \verb|instr ; try instr catch instr ; try instr | & \verb|$ I0| & \textsc{shift} \\
                    \verb|catch instr finally instr ; instr $| & &  \\ 
                    & & \\ \hline

                    \verb|; try instr catch instr ; try instr catch| & \verb|$ I0 I1| & \textsc{reduce}\\
                    \verb|instr finally instr ; instr $| & & \verb|(4)| \\ 
                    & & \\ \hline

                    \verb|; try instr catch instr ; try instr catch| & \verb|$ I0 I2| & \textsc{shift} \\
                    \verb|instr finally instr ; instr $| & & \\ 
                    & & \\ \hline

                    \verb|try instr catch instr ; try instr catch| & \verb|$ I0 I2 I4| & \textsc{shift}\\
                    \verb|instr finally instr ; instr $| & &  \\ 
                    & & \\ \hline

                    \verb|instr catch instr ; try instr catch instr| & \verb|$ I0 I2 I4 I3| & \textsc{shift} \\
                    \verb|finally instr ; instr $| & &  \\ 
                    & & \\ \hline

                    \verb|catch instr ; try instr catch instr| & \verb|$ I0 I2 I4 I3 I1| & \textsc{reduce} \\
                    \verb|finally instr ; instr $| & & \verb|(4)| \\ 
                    & & \\ \hline

                    \verb|catch instr ; try instr catch instr| & \verb|$ I0 I2 I4 I3 I5| & \textsc{shift} \\
                    \verb|finally instr ; instr $| & & \\ 
                    & & \\ \hline

                    \verb|instr ; try instr catch instr finally| & \verb|$ I0 I2 I4 I3 I5| & \textsc{shift} \\
                    \verb|instr ; instr $| & \verb|I7| & \\ 
                    & & \\ \hline

                    \verb|; try instr catch instr finally instr ;| & \verb|$ I0 I2 I4 I3 I5| & \textsc{reduce} \\
                    \verb|instr $| & \verb|I7 I1| & \verb|(4)| \\ 
                    & & \\ \hline

                    \verb|; try instr catch instr finally instr ;| & \verb|$ I0 I2 I4 I3 I5| & \textsc{shift} \\
                    \verb|instr $| & \verb|I7 I8| & \\ 
                    & & \\ \hline
                \end{tabular}
            \end{center}

            \begin{center}
                \begin{tabular}{ | l | l | c |}
                    \hline
                    \verb|try instr catch instr finally instr ;| & \verb|$ I0 I2 I4 I3 I5| & \textsc{shift} \\
                    \verb|instr $| & \verb|I7 I8 I4| & \\ 
                    & & \\ \hline

                    \verb|instr catch instr finally instr ; instr $| & \verb|$ I0 I2 I4 I3 I5| & \textsc{shift} \\
                    \verb|                                          | & \verb|I7 I8 I4 I3| & \\ 
                    & & \\ \hline

                    \verb|catch instr finally instr ; instr $| & \verb|$ I0 I2 I4 I3 I5| & \textsc{reduce} \\
                    & \verb|I7 I8 I4 I3 I1| & \verb|(4)| \\ 
                    & & \\ \hline

                    \verb|catch instr finally instr ; instr $| & \verb|$ I0 I2 I4 I3 I5| & \textsc{shift} \\
                    & \verb|I7 I8 I4 I3 I5| & \\ 
                    & & \\ \hline

                    \verb|instr finally instr ; instr $| & \verb|$ I0 I2 I4 I3 I5| & \textsc{shift} \\
                    & \verb|I7 I8 I4 I3 I5 I7| & \\ 
                    & & \\ \hline

                    \verb|finally instr ; instr $| & \verb|$ I0 I2 I4 I3 I5| & \textsc{reduce} \\
                    & \verb|I7 I8 I4 I3 I5 I7| & \verb|(4)| \\
                    & \verb|I1| & \\ \hline

                    \verb|finally instr ; instr $| & \verb|$ I0 I2 I4 I3 I5| & \textsc{shift} \\
                    & \verb|I7 I8 I4 I3 I5 I7| & \\
                    & \verb|I8| & \\ \hline

                    \verb|instr ; instr $| & \verb|$ I0 I2 I4 I3 I5| & \textsc{shift} \\
                    & \verb|I7 I8 I4 I3 I5 I7| & \\
                    & \verb|I8 I9| & \\ \hline

                    \verb|; instr $| & \verb|$ I0 I2 I4 I3 I5| & \textsc{reduce} \\
                    & \verb|I7 I8 I4 I3 I5 I7| & \verb|(4)|\\
                    & \verb|I8 I9 I1| & \\ \hline

                    \verb|; instr $| & \verb|$ I0 I2 I4 I3 I5| & \textsc{reduce} \\
                    & \verb|I7 I8 I4 I3 I5 I7| & \verb|(1)|\\
                    & \verb|I8 I9 I10| & \\ \hline

                    \verb|; instr $| & \verb|$ I0 I2 I4 I3 I5| & \textsc{reduce} \\
                    & \verb|I7 I8 I4 I6| & \verb|(3)| \\
                    & & \\ \hline

                    \verb|; instr $| & \verb|$ I0 I2 I4 I3 I5| & \textsc{shift} \\
                    & \verb|I7 I8| &  \\
                    & & \\ \hline

                    \verb|instr $| & \verb|$ I0 I2 I4 I3 I5| & \textsc{shift} \\
                    & \verb|I7 I8 I4| &  \\
                    & & \\ \hline

                    \verb|$| & \verb|$ I0 I2 I4 I3 I5| & \textsc{reduce} \\
                    & \verb|I7 I8 I4 I1| &  \verb|(4)|\\
                    & & \\ \hline

                    \verb|$| & \verb|$ I0 I2 I4 I3 I5| & \textsc{reduce} \\
                    & \verb|I7 I8| &  \verb|(2)|\\
                    & & \\ \hline

                    \verb|$| & \verb|$ I0 I2 I4 I6| & \textsc{reduce} \\
                    &  &  \verb|(3)|\\
                    & & \\ \hline

                    \verb|$| & \verb|$ I0 I2| & \textsc{shift} \\
                    &  & \\
                    & & \\ \hline

                    & \verb|$ I0 I2 I11| & \textsc{accept} \\
                    &  &  \\
                    & & \\ \hline
                \end{tabular}
            \end{center}

            \newpage

            \item \textit{(1.5 puntos)} Calcule las funciones de precedencia \verb|f| y \verb|g| 
            seg\'un el algoritmo estudiado en clase (o argumente por qu\'e dichas funciones no 
            pueden ser construidas).

            \textbf{Respuesta:} Representaremos el grafo colocando cada nodo (clase de equivalencia)
            apuntando al conjunto de sus nodos sucesores:\\

            \begin{center}
                \begin{tabular}{ | l c l | }
                    \hline
                    \textsc{NODO}                   &                     & \textsc{SUCESORES} \\
                    \hline
                    $\{G_{instr}\}$                 &  $\Longrightarrow$  & $\{ \{F_{try}, G_{catch}\}, \{F_{\$}\}, \{F_{catch}, G_{finally}\}, \{F_{finally}\}, \{F_{;}\} \}$ \\
                    $\{G_{try}\}$                   &  $\Longrightarrow$  & $\{ \{F_{try}, G_{catch}\}, \{F_{\$}\}, \{F_{catch}, G_{finally}\}, \{F_{finally}\}, \{F_{;}\} \}$ \\
    
                    & & \\ \hline
                    $\{F_{instr}\}$                 &  $\Longrightarrow$  & $\{ \{F_{try}, G_{catch}\},  \{G_{\$}\}, \{F_{catch}, G_{finally}\}, \{G_{;}\}\}$ \\
                    $\{F_{;}\}$                     &  $\Longrightarrow$  & $\{ \{G_{\$}\}, \{G_{;}\}\}$ \\
                    $\{F_{finally}\}$               &  $\Longrightarrow$  & $\{ \{F_{try}, G_{catch}\},  \{G_{\$}\}, \{F_{catch}, G_{finally}\}, \{G_{;}\}\}$ \\
    
                    & & \\ \hline
                    $\{G_{;}\}$                     &  $\Longrightarrow$  & $\{ \{F_{catch}, G_{finally}\}, \{F_\$ \} \}$ \\
    
                    & & \\ \hline
                    $\{F_{catch}, G_{finally}\}$    &  $\Longrightarrow$  & $\{ \{F_{try}, G_{catch}\}, \{G_{\$}\} \}$ \\
    
                    & & \\ \hline
                    $\{F_{\$}\}$                    &  $\Longrightarrow$  & $\emptyset$ \\
                    $\{G_{\$}\}$                    &  $\Longrightarrow$  & $\emptyset$ \\
                    $\{F_{try}, G_{catch}\}$        &  $\Longrightarrow$  & $\emptyset$ \\
                    \hline
                \end{tabular}
            \end{center}

            As\'i, los valores de \verb|f| y \verb|g| son: \\

            \begin{center}
                \begin{tabular}{ | c | c | c | c | c | c | c |}
                    \hline
                    & \verb|try| & \verb|catch| & \verb|finally| & \verb| ; | & \verb|instr| & \verb| $ | \\ \hline
                    \verb| f | & 0 & 1 & 3 & 3 & 3 & 0 \\ \hline
                    \verb| g | & 4 & 0 & 1 & 2 & 4 & 0 \\ \hline 
                \end{tabular}
            \end{center}
            \vspace*{1,5cm}
        \end{enumerate}

        \item \textit{(4 puntos)} Considererando la misma gram\'atica de las preguntas anteriores, 
        implementaremos una sem\'antica para este lenguaje donde las instrucciones tienen un valor 
        adem\'as del efecto de borde que ocasionan. Para cualquier instrucción, su valor ser\'a el 
        valor de la \'ultima expresi\'on que haya sido evaluada (o de la \'ultima instrucci\'on 
        ejecutada, equivalentemente).

        \begin{enumerate}
            \item \textit{(2 puntos)} Aumente el s\'imbolo no-terminal \verb|I| con un atributo 
            \verb|tipo|, que contenga el tipo del valor que retorna la instrucci\'on representada 
            en \verb|I|. Puede suponer que cuenta con un tipo \verb|Either A B| para representar 
            un tipo que es opcionalmente \verb|A| o \verb|B|. Puede suponer, adem\'as, que el 
            sı\'imbolo \verb|instr| tiene un atributo intr\'inseco tipo que tiene el tipo para ese 
            terminal. Puede agregar todos los atributos adicionales que desee a \verb|I|, cuidando 
            que la gram\'atica resultante sea \textit{S-atribuida}.

            \textbf{Respuesta:}
            \begin{verbatim}
S   ->  I $                         { S.tipo <- I.tipo }  
I   ->  try I0 catch I1 finally I2  { I.tipo <- I2.tipo }
     |  try I0 catch I1             { I.tipo <- Either I0.tipo I1.tipo }
     |  I0 ; I1                     { I.tipo <- I1.tipo }
     |  instr                       { I.tipo <- instr.tipo }
            \end{verbatim}
            \newpage

            \item \textit{(1 punto)} Tenemos a nuestra disposici\'on un reconocedor descendente. 
            La gram\'atica anterior tiene prefijos comunes y recursi\'on izquierda. Transforme la 
            gram\'atica de tal forma que sea apropiada para un reconocedor descendente. Recuerde 
            agregar atributos y reglas de tal forma que a\'un se calcule el tipo de la instrucci\'on
            en tipo, cuidando que la gram\'atica resultante sea L–atribuida.
            
            \textbf{Respuesta:} 

            \begin{verbatim}
S   ->  I { S.tipo <- I.tipo } $  

I   ->  E { R.in <- E.tipo } R { I.tipo <- R.tipo }   

R   ->  ; I { R0.in <- I.tipo } R0 { R.tipo <- R0.tipo }
     |  { R.tipo <- R.in }

E   ->  try I0 catch { I1.in <- I0.tipo } 
            I1 { F.in <- Either I1.in I1.tipo } 
            F { E.tipo <- F.tipo }   
     |  instr { E.tipo <- instr.tipo } 
     
F   ->  finally E { F.tipo <- E.tipo }       
     |  { F.tipo <- F.in }
            \end{verbatim}
            \vspace*{1cm}
        
            \item \textit{(1 punto)} Construya un reconocedor recursivo descendente a partir 
            de su gram\'atica. Esto es, escriba las funciones (en el lenguaje de su elección) 
            que reconozca frases en el lenguaje y calculen el atributo tipo para una instrucci\'on 
            bien formada. Deben llevar una variable \textit{lookahead} que contenga el siguiente 
            sı\'imbolo de la entrada en todo momento. Su programa debe funcionar correctamente 
            para cualquier entrada y estar alojada en un repositorio \textit{git} p\'ublico. \\

            \textbf{Respuesta:} La implementaci\'on del reconocedor recursivo descendente se 
            encuentra en el siguiente 
            \href{https://github.com/ArriagaAmin/LenguajesIIParcial1/blob/main/Pregunta2/TryCatchLector}{archivo} 
            del repositorio. Debe correr el script con \verb|python3|, el cual le proporcionar\'a
            un prompt para usar el reconocedor, cuya sintaxis es:
            
            \begin{verbatim}
                            $> PARSE [<string> ...]
                            $> SALIR
            \end{verbatim}

        \end{enumerate}

        \item \textit{(5 puntos)} Se desea que modele e implemente, en el lenguaje de su elecci\'on, 
        un generador de analizadores sint\'acticos para gram\'aticas de operadores. Investigue 
        herramientas para pruebas unitarias y cobertura en su lenguaje escogido y agregue pruebas a 
        su programa que permitan corroborar su correcto funcionamiento. Como regla general, su programa 
        deber\'ia tener una cobertura (de l\'ineas de c\'odigo y de bifuraci\'on) mayor al 80\%. \\

        \textbf{Respuesta:} La implementaci\'on del generador de analizadores sint\'acticos se 
        encuentra en el siguiente 
        \href{https://github.com/ArriagaAmin/LenguajesIIParcial1/blob/main/Pregunta2/TryCatchLector}{archivo} 
        del repositorio.
    \end{enumerate}

\end{document}